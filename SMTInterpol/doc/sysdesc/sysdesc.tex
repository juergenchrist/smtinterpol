\documentclass[a4paper]{easychair}
\usepackage[english]{babel}
\usepackage{xspace}
\usepackage{hyperref}
\usepackage{bashful}

\newcommand\SI{SMTInterpol\xspace}
\newcommand\SIrem{SMTInterpol-remus\xspace}
\newcommand\smtlib[1]{\texttt{#1}}
\newcommand\prf{\mathit{prf}}
\newcommand{\version}{\splice{git describe}}
\newcommand{\TODO}[1]{\textcolor{red}{#1}}

\title{\SI\\{\large Version \version}}

\author{Max Barth\inst{1} \and Jochen Hoenicke\inst{2} \and Tanja Schindler\inst{3}}
\institute{
  Ludwig Maximilian University of Munich, \email{max.barth@lmu.de} \and
  Certora, \email{jochen@certora.com} \and
  University of Basel, \email{tanja.schindler@unibas.ch}
}
\titlerunning{\SI \version}
\authorrunning{Barth, Hoenicke and Schindler}

\begin{document}
\maketitle
\section*{Description}
\SI is an SMT solver written in Java and available under LGPL v3.
It supports the combination of the theories of uninterpreted functions, arithmetic over integers and reals, arrays, and datatypes.
Furthermore it can produce models, proofs, unsatisfiable cores, and interpolants.
The solver reads input in SMT-LIB format.
It includes parsers for DIMACS, AIGER, and SMT-LIB version 2.6.

The solver is based on the well-known DPLL(T)/CDCL framework~\cite{DBLP:conf/cav/GanzingerHNOT04}.
It uses variants of standard algorithms for CNF conversion~\cite{DBLP:journals/jsc/PlaistedG86} and congruence closure~\cite{DBLP:conf/rta/NieuwenhuisO05}.
The solver for linear arithmetic is based on Simplex~\cite{DBLP:conf/cav/DutertreM06}, the sum-of-infeasibility algorithm~\cite{DBLP:conf/fmcad/KingBD13}, and branch-and-cut for integer arithmetic~\cite{DBLP:conf/cav/ChristH15,DBLP:conf/cav/DilligDA09}.
The array decision procedure is based on weak equivalences~\cite{DBLP:conf/frocos/ChristH15} and includes an extension for constant arrays~\cite{DBLP:conf/vmcai/HoenickeS19}.
The decision procedure for data types is based on the rules presented in~\cite{DBLP:journals/jsat/BarrettST07}.
The solver for quantified formulas performs an incremental search for conflicting and unit-propagating instances of quantified formulas~\cite{DBLP:conf/vmcai/HoenickeS21} which is complemented with a version of enumerative instantiation~\cite{DBLP:conf/tacas/ReynoldsBF18} to ensure completeness for the finite almost uninterpreted fragment~\cite{DBLP:conf/cav/GeM09}.
Theory combination is performed based on partial models produced by the theory solvers~\cite{DBLP:journals/entcs/MouraB08}.
\SI comes with a proof production~\cite{DBLP:conf/smt/HoenickeS22} that is based on resolution using a minimal set of axioms.

The main focus of \SI is the incremental track.
This track simulates the typical application of \SI where a user asks multiple queries.
As an extension, \SI supports interpolation for all supported theories~\cite{DBLP:journals/jar/ChristH16,DBLP:conf/cade/HoenickeS18,DBLP:conf/vmcai/HoenickeS19,DBLP:conf/smt/HenkelHS21}.  It computes quantifier-free interpolants for quantifier-free input formulas.
This makes it useful as a backend for software verification tools.
In particular, \textsc{Ultimate Automizer}\footnote{\url{https://ultimate.informatik.uni-freiburg.de/}} and \textsc{CPAchecker}\footnote{\url{https://cpachecker.sosy-lab.org/}}, the winners of the SV-COMP 2016--2023, used \SI.

\SI aims for producing witnesses (proofs and models).  The competition version will check these witnesses before reporting the result to ensure soundness.  Only sat results for quantified logics are not checked.

\section*{Competition Version}
The version of \SI submitted to the SMT-COMP 2024 contains
preliminary support for non-linear benchmarks.  Multiplication of arbitrary terms is handled by the pre-processor.  However, the linear solver will treat products of variables as unconstrained variables.  If a model is found, the solver reports unknown if a non-linear multiplication is part of the tableau.

We also have an experimental bit-vector solver based on a translation to integer arithmetic.  The bitvector solver does not yet support bit-wise operations (like bvand/bvor/bvxor). It also has no support for proof or models, so there is no extra check to ensure soundness for these theories.

\section*{Authors, Logics, and Tracks}
The code was written by Max Barth, Leon Cacace, J{\"u}rgen Christ, Daniel Dietsch, Leonard Fichtner, Joanna Greulich, Elisabeth Henkel, Matthias Heizmann, Jochen Hoenicke, Moritz Mohr, Alexander Nutz, Markus Pomrehn, Pascal Raiola, and Tanja Schindler.
Further information about \SI can be found at
\begin{center}
  \url{https://ultimate.informatik.uni-freiburg.de/smtinterpol/}
\end{center}
The sources are available via GitHub
\begin{center}
  \url{https://github.com/ultimate-pa/smtinterpol}
\end{center}

\SI participates in the single query track, the incremental track, the unsat core track, and the model validation track.
It supports all combinations of uninterpreted functions, arithmetic, arrays, datatypes, and quantified formulas.
However, the model generator currently does not support quantifiers.

In the single query track, the incremental track, the unsat core track \SI participates in the logics matched by
\verb!(QF_)?(AX?)?(UF)?(DT)?([IR]DL|[NL][IR]*A)?!\footnote{\SI will mostly ignore non-linear arithmetic, and report unknown on benchmarks that need non-linear reasoning.  It fully supports modulo and division by constants, though.},
and in the model validation track in the logics matched by
\verb!(QF_)(AX?)?(UF)?(DT)?([IR]DL|[NL][IR]*A)?!.

\bibliography{sysdesc}
\bibliographystyle{alpha}
\end{document}

%%  LocalWords:  parsers SMT LGPL datatypes unsatisfiable DIMACS DPLL
%%  LocalWords:  interpolants AIGER CDCL CNF infeasibility backend SV
%%  LocalWords:  enumerative Automizer CPAchecker logics unsat Bool
%%  LocalWords:  unsatisfiability subproof subproofs arity AST farkas
%%  LocalWords:  subterms typecheck Webpage
