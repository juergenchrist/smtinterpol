\documentclass[a4paper]{article}

\usepackage[ngerman]{babel}
\usepackage[utf8]{inputenc}
\usepackage{mathpartir}
\usepackage{pxfonts}

\title{}
\author{Markus Pomrehn}

\begin{document}

\maketitle

\abstract{}

\section{Einführung}

\section{Beweisformat}
\subsection{Alt}
\subsubsection{Vorhandene Beweisregeln}


\paragraph{expand:} SMTInterpol erweitert assoziative Funktionen, je nach dem ob sie links- oder rechts-assoziativ.
Aus $f(t_1,t_2...,t_{n-1},t_n)$ wird bei links-asso\-zia\-ti\-ven Funktionen $f(f(...f(t_1,t_2),...),t_n)$ und bei rechts-assoziativen Funktionen $f(t_1,f(...,f(t_{n-1},t_n)))$.
Außerdem wird bei verkettbaren Funktionen aus $f(t_1,t_2,t_3...,t_{n-1},t_n)$ = $f(t_1,t_2) \land f(t_2,t_3) \land ... \land f(t_{n-1},t_n)$.

\paragraph{expandDef:} SMTLib ermöglicht es Benutzern eigene Funktionen zu definieren.
Diese Funktion ersetzt n durch ihre Defintition d.
\[
\inferrule*[left=ExpandDef]{ }{n(t) = d[t]}
\]
\paragraph{trueNotFalse:} Gleichheit von Boolschen Termen, mit einem Term der falsch ist und einem der wahr ist, vereinfache zu falsch.
\begin{mathpar}
  \inferrule*[left=TrueNotFalse,right={$\exists j,k\in I.\ t_j=true \land
      t_k=false$}]{ } {(=_{i\in I}\ t_i) = false}
      \end{mathpar}
\paragraph{constDiff:} Gleichheit unterschiedlicher Konstanten ist falsch.
\begin{mathpar}
\inferrule*[left=ConstDiff,right={$\exists j,k\in I.\ t_j=c_j \land
      t_k=c_k\land c_j\neq c_k$}]{ } {(=_{i\in I}\ t_i) = false}
\end{mathpar}
\paragraph{eqTrue:} Gleicheit zu wahr, wird durch ein Konjunkt ersetzt.
\begin{mathpar}
\inferrule*[left=EqTrue,right={$\exists j\in I.\ t_j=true\land I'\subset
      I\land j\not\in I'$}]{ } {(=_{i\in I}\ t_i) = (and_{i'\in I'}\ t_{i'})}
\end{mathpar}
\paragraph{eqFalse:} Gleichheit zu falsch, wird durch die Negation der Veroderung der zu falsch gleichen Terme ersetzt.
\begin{mathpar}
\inferrule*[left=EqFalse,right={$\exists j\in I.\ t_j=false\land I'\subset
      I\land j\not\in I'$}]{ } {(=_{i\in I}\ t_i) = (not\ (or_{i'\in I'}\
    t_{i'}))}
\end{mathpar}
\paragraph{eqSame:} Wenn alle Terme gleich sind, vereinfache zu wahr.
\begin{mathpar}
\inferrule*[left=EqSame,right={$\forall i,j\in I.\ t_i\equiv t_j$}]{
  }{(=_{i\in I}\ t_i) = true}
\end{mathpar}
\paragraph{eqSimp:} Ersetze gleiche Terme durch diesen Terme.
\begin{mathpar}
\inferrule*[left=EqSimp,right={$I'\subset I\land |I'| = |\{t_i.\ i\in
      I\}|\land\{t_i.\ i\in I\} = \{t_j.\ j\in I'\} $}]{ }{(=_{i\in I}\ t_i) =
    (=_{i'\in I'}\ t_{i'})}
\end{mathpar}
\paragraph{eqBinary}
\paragraph{distinctBool}
\paragraph{distinctSame}
\paragraph{distinctNeg}
\paragraph{distinctTrue}
\paragraph{distinctFalse}
\paragraph{distinctBoolEq}
\begin{mathpar}
\inferrule*[left=DistinctBoolEq,right={$sort(t_0)=sort(t_1)=Bool\land
      ((t_0',t_1')=(\lnot t_0,t_1)\lor(t_0',t_1')=(t_0,\lnot t_1))$}]{ }
  {(distinct\ t_0\ t_1) = (=\ t_0'\ t_1')}
\end{mathpar}
\paragraph{distinctBinary}
\paragraph{notSimp}
\paragraph{orSimp}
\paragraph{orTaut}
\paragraph{iteTrue}
\paragraph{iteFalse}
\paragraph{iteSame}
\paragraph{iteBool1}
\paragraph{iteBool2}
\paragraph{iteBool3}
\paragraph{iteBool4}
\paragraph{iteBool5}
\paragraph{iteBool6}
\paragraph{andToOr}
\paragraph{xorToDistinct}
\paragraph{impToOr}
\paragraph{canonicalSum}
\paragraph{leqToLeq0}
\paragraph{ltToLeq0}
\paragraph{geqToLeq0}
\paragraph{gtToLeq0}
\paragraph{leqTrue}
\paragraph{leqFalse}
\paragraph{desugar}
\paragraph{divisible}
\paragraph{modulo}
\paragraph{modulo1}
\paragraph{modulo-1}
\paragraph{moduloConst}
\paragraph{div1}
\paragraph{div-1}
\paragraph{divConst}
\paragraph{toInt}
\paragraph{toReal}
\paragraph{flatten}
\paragraph{strip}
\paragraph{storeOverStore}
\paragraph{selectOverStore}
\paragraph{storeRewrite}

Probleme:

\subsection{Neu}

\section{Entwurf}

Wie wird ein Beweis generiert?
Beispiel Formel und der Transformationsbeweis

\section{Implementation}

Alt:
NonRecursive beschreiben.
Klassen beschreiben: ProofTracker, TermCompiler, Clausifier

Neu:
Änderungen an ProofTracker/TermCompiler, Clausifier beschreiben.
Neue Interaktion und Interface zwischen den Klassen beschreiben.


\section{Verwandte Arbeiten}

\section{Fazit und Ausblick}

\bibliography{references}

\end{document}
